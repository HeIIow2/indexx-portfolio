\documentclass{article} 

\usepackage[margin=1in]{geometry}
\usepackage{german}
\usepackage{ mathtools, amssymb, amsthm }
\usepackage{graphicx}
\usepackage{color}
\usepackage{siunitx}

\begin{document}

\section*{Andere Formeln}

\subsection*{Massenwirkungsgesetzt}

$K_c = \frac{c(C)^c \cdot c(D)^d} {c(A)^a \cdot c(B)^b}$

\subsection*{pH-Wert}

$pH = -\log{H_3O^+}$

$pH = \frac{1}{2} \cdot (-\log{HA})$

\subsection*{$pK_S$ Wert}

$K_S = \frac {c(H_3O^+) \cdot c(A)} {c(HA)}$

$pK_S = -\log{K_S}$

\section*{Titrationskurve}

Bei der Titration hat man eine Lößung mit unbekannter Konzentration. Man hat eine Probelösung mit bekannter Konzentration. In der unbekannten Lösung ist ein Indikator. Dann macht man die Probelößung in die andere Lösung bis es neutral ist und man kann die Konzentration bestimmen




\section*{Pufferlösungen}

\subsection*{Was und wie}

Eine Pufferlößung ist eine Lösung eines konjungiertes Säure-Base-Paars, das ihren pH Wert nach Zugabe einer Säure bzw. Base nicht bzw. kaum ändert.

\subsection*{Hendersen Hasselbalg}

Der pH Wert ist abhängig von:

\begin{itemize}
\item Dem $pk_S$ Wert
\item Dem Verhältnis von $A^-$ und $HA$ Ionen
\end{itemize}

$pH = pk_S + \lg{\frac{(A^-)} {c(HA)}}$

$pk_S = pH - \lg{\frac{c(A^-)} {c(HA)}}$

\section*{Fischer Projektion}

\subsection*{D-Glukose}

Ta Tü Ta Ta 

\includegraphics*{glukose.png}

\subsection*{Asymmetrische C-Atome}

Asymmetrische C-Atome sind Kolenstoff Atome die an vier anderen Atomen anbinden.

\subsection*{Isomerie}
\begin{itemize}
\item Enantiomere: Spiegelzucker (D-L Glukose)
\item Diastereomere: Gleich aufgebaut aber verhalten sich nicht wie Bild und Spiegelbild.
\item Stereoisomere: Wenn die Atome unterschiedlich aufgebaut sind aber sich anders verhalten (ähnlich wie diastereomere)
\end{itemize}

\subsection*{Optische Aktivität}

Optische Aktivität dreht das Licht. Das Racemat, 1:1 Verhältnis aus zwei Emanziomere, hebt die optische Aktivität auf.

\end{document}