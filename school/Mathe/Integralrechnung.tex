\documentclass{article} 

\usepackage[margin=1in]{geometry}
\usepackage{german}
\usepackage{ mathtools, amssymb, amsthm }
\usepackage{graphicx}
\usepackage{color}
\usepackage{siunitx}

\begin{document}

\section*{Integralrechnung}

\subsection*{Stammfunktionen}

$f(x) = x^5$
\newline
$f'(x) = 5x^4$
\newline
\newline
$F(x) = \frac{1}{101}x^{101} = \frac{x^{101}}{101}$
\newline
$f(x) = x^{100}$
\newline
\newline

\noindent \textbf{Beobachtung:} wenn $f'(x) = x^3$ gilt, dann könnte $f(x) = \frac{1}{4}x^4$ gelten, es könnte aber auch $f(x) = \frac{1}{4}x^4 + 4$ oder $f(x) = \frac{1}{4}x^4 + 367$ gelten.
\vspace{0.5cm}

\noindent \textbf{Def:} $F(x)$ heißt Stammfunktion fon $f(x)$, falls $F'(x) = f(x)$ gilt.
\vspace{0.5cm}

\noindent \textbf{Bemerkung:} Wer F eine Stammfunktion von f ist, dann ist auch G mit $G(x) = F(x) + c$ für $c \widehat{=} R$ eine Stammfunktion von $f$.
\vspace{0.5cm}

\noindent \textbf{Beweis:} 

$G'(x) = F'(x) + 0$  

$G'(x) = f(x)$
\vspace{0.5cm}

\noindent \textbf{Aufgabentypen}

\begin{itemize}
\item Geben Sie 3 verschiedene Stammfunktionen von $f(x) = 5x^3 - 7x^2 + 8x + 4$ an!
\item Geben Sie alle Stammfunktionen von f aus 1) an! \newline $F(x) = \frac{5}{4} x^4 - \frac{7}{3}x^3 + 4x^2 + 4x + c; c \widehat{=} \mathbb{R}$
\item Geben die \textbf{die} Stammfunktion von f aus 1) an, für die $F(1) = 100$ gilt!
\end{itemize}

\section*{Integralrechnung}

Eines der Ziele der Integralrechnung ist die exakte Berechnung von Flächen, die deurch Funktionsgraphen umrandet werden. Die ersten Ideen dazu sind SEHR alt, so stammt die folgene Einschachtel-Idee schon aus der Antike, z.B. bei Archimeedes (ca. 285 - 212) zu finden (Trapeze statt Rechtecke).

\hrule

\subsection*{Formel}

$A = \int_{0}^{1} x^2 \cdot dx = [\frac{1}{3} x^3]_0^1 = \frac{1}{3} \cdot 1^3 -\frac{1}{3} \cdot 0^3 = \frac{1}{3} - 0 = \frac{1}{3}$

\begin{itemize}
\item Man muss die Stammfunktion (Aufleitung) der funktion hinter $\int$ bekommen
\item In diese setzt man die obere Integralzahl ein.
\item Dann die untere.
\item Dann substrahiert man das ergebnis des oberen Ergebnisses mit dem Ergebniss des unteren.
\end{itemize}

\end{document}