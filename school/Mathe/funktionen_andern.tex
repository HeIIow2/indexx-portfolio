\documentclass{article} 

\usepackage[margin=1in]{geometry}
\usepackage{german}
\usepackage{ mathtools, amssymb, amsthm }
\usepackage{graphicx}
\usepackage{color}
\usepackage{siunitx}

\begin{document}

\section*{Strecken}

\subsection*{in y-Richtung}

$g(x) = Streckfaktor \cdot f(x)$
\newline
\textbf{Bemerkung:} Wenn der Faktor negativ ist,  dann bekommt man eine Kombination aus Streckung mit positivem Faktor und eine Spiegelung an der x-Achse.

\subsection*{in x-Richtung}

$g(x) = f(\frac{1} {Streckfaktor} \cdot x)$
\newline
$g(x) = f(\frac{x} {Streckfaktor})$
\newline
\textbf{Bemerkung:} Wenn der Faktor negativ ist, bekommt man eine Kombination aus einer Streckung in x-Richtung mi dem Normalisierten Faktor und einer Spiegelung an der y-Achse.

\end{document}