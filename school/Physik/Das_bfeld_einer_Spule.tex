\documentclass{article} 

\usepackage[margin=1in]{geometry}
\usepackage{german}
\usepackage{ mathtools, amssymb, amsthm }
\usepackage{graphicx}
\usepackage{color}
\usepackage{siunitx}

\begin{document}

\section*{Das B-Feld einer Spule}

\includegraphics{Erklärung-Feld-einer-spule.png}

\begin{itemize}
\item Im inneren der Spule überlagern sich die B-Felder gleichsinnig und verstärken sich. (konstruktive Überlagerung)
\item Zwischehn den Drähten heben sich die Felder gegenseitig auf (destruktive Überlagerung)
\item Auserhalb der Spulen schächen sich die Felder gegenseitig ab.
\end{itemize}

\includegraphics[width=.5\textwidth]{b-Feld-einer-Spule.png}

\begin{itemize}
\item Im inneren der Spule sind die Feldlinien parallel $\rightarrow$ das Feld ist homogen.
\item Außerhalb der Spule entsteht ein Feldlinienbild ähnlich eines Stabmagneten
\end{itemize}

\end{document}