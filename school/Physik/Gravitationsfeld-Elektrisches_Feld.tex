\documentclass{article} 

\usepackage[margin=1in]{geometry}
\usepackage{german}
\usepackage{ mathtools, amssymb, amsthm }
\usepackage{graphicx}
\usepackage{color}
\usepackage{siunitx}

\begin{document}

\section*{Vergleich Gravitationsfeld - Elektrisches Feld \newline (Analogie-Betrachtung Mechanik - Elektrik)}

\subsection*{Gravitationsgesetz (Newton)}

\begin{itemize}
\item Massen üben gegenseitig Anziehungskräfte aus
\item Gravitationsgesetz: $F_{Grav} = \Gamma \cdot \frac{M \cdot m} {r^2}$
\item Befinden sich zwei Körper gleicher Masse auf der gleichen Höhenlinie, erfahren sie die gleiche Gewichtskraft.
\end{itemize}

\subsection*{Elektrik}

\begin{itemize}
\item unterschiedliche Ladungen üben Anziehungskräfte aus, gleiche stoßen sich ab.
\item Coulombgesetz: $F_C = k \cdot \frac{q_1 - q_2} {r^2}$
\item Befinden sich zwei Körper gleicher Ladung auf der gleichen Aquipotenziallinie erfahren sie die gleiche elektrische Feldkraft.
\end{itemize}

\end{document}