\documentclass{article} 

\usepackage[margin=1in]{geometry}
\usepackage{german}
\usepackage{ mathtools, amssymb, amsthm }
\usepackage{graphicx}
\usepackage{color}
\usepackage{siunitx}

\begin{document}

\section*{Die Lorenzkraft $F_{L-}$}

(Die Kraft auf einen stromdurchflossenen Leiter im B-Feld)

\includegraphics{lorentzkraft.png}

Diese Kraft die die Schaukel bewegt steht sowohl zum B-Fld als auch zur Richtung der Stromstärke \textbf{immer} senkrecht.
\newline
Durch die Überlagerung der B-Felder des Magneten und des stromdurchflossenen Leiters wird das resultierende B-Feld auf einer Seite verstärkt, auf der anderen abgeschwächt.

\subsection{Bestimmung der Richtung von $F_L$ mit der 3-Finger Regel (UVW)-Regel}

\begin{itemize}
\item U Ursache: Richtung der Stromstärke
\item V Vermittlung: Richtung des B-Feldes
\item W Wirkung: Richtung der Lorenzkraft
\end{itemize}

\includegraphics{dreifinger.png}

\end{document}