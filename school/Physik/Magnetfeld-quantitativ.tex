\documentclass{article} 

\usepackage[margin=1in]{geometry}
\usepackage{german}
\usepackage{ mathtools, amssymb, amsthm }
\usepackage{graphicx}
\usepackage{color}
\usepackage{siunitx}

\begin{document}

\section*{Das Magnetfeld (B-Feld) Quantitativ}

\subsection*{Stabmagnet}

\includegraphics{stabmagent-feldlinien.png}

\subsection*{Hufeisenmagnet}

\includegraphics{hufeisen-feldlinien.png}

\subsection*{Eigenschaften Magnetischer Feldlinien}

\begin{itemize}
\item sie schneiden sich nicht
\item verlaufen sie parallel, so nennt man das Feld homogen
\item sie verlaufen außerhalb des Magneten von Nord nach Süd. per Definition: Die Richtung der Magnetischen Feldlinien ist die Richtung, in die sich der Nordpol einer dort frei beweglich aufgestellten Magnetnadel einstellt.
\item Die Richtung der magnetischen Kraft erhält man durch die Tangente an der Feldlinie
\item Um einen Stromdurchflossenen Leiter verlaufen die Feldlininen in konzentrischen Kreisen
\item Die Anzahl der Feldlinien pro Fläche ist ein Maß für die Stärke des B-Feldes
\item Magnetische Feldlinien sind in sich geschlossen
\end{itemize}

\subsection*{Feldlinie zwischen 2 Stabmagneten}

\includegraphics{stabmagneten-feldlinien.png}

\subsection*{1820 Versuch von Oersted}

Magnetnadel in der Nähe eines Stromdurchfloddenen Leiters

\includegraphics{oerstod.jpg}

\noindent Beobachtung: Je nach Position der Magnetnadel richtet dese sich am Leiter aus.

\noindent Erklärung: Um den stromdurchflossenen Leiter muss ein Magnetfeld entstandenn sein.

Darstellung der Richtung enes Vektoren mit hilfe eines Pfeil mit schweifs

\begin{itemize}
\item Kreis mi Punkt: Kommt zu einem
\item Kreis mit Kreuz: geht du einem
\end{itemize}

\subsection*{Merke}

Um einen Stromdurchflossenen Leiter entsteht ein Konzentrisches MAgnetfeld mit in sich geschlossenen Feldlinien.
Die Richtung der Feldlinien ermittelt man mit der Rechten hand Regel: Zeigt der Daumen der Rechten Hand in Richtung der technischen Stromstärke, so geben die gekrümmten Finger die Richtung der Feldliene an.

\subsection*{Merke 2}

In zukunft verwenden wir die Rechte Hand für die Richtung der Technischen Stromstärke (von + nach -) oder für die Bewegungsrichtung von positiven Ladungsträgern, und die linke Hand für die Physikalische Stromrichtung (von - nach +) oder für die Bewegungsrichtung von negativen Ladungsträgern.

\end{document}