\documentclass{article} 

\usepackage[margin=1in]{geometry}
\usepackage{german}
\usepackage{ mathtools, amssymb, amsthm }
\usepackage{graphicx}
\usepackage{color}
\usepackage{siunitx}
\usepackage{wrapfig}

\begin{document}

\section*{Das Magnetfeld}

Magnete besitzen einen Nord- und Südpol.

\includegraphics[width=0.5\textwidth]{nord-südpol.png}

\noindent Gleichnamige Pole stoßen sich ab, ungleichnamige ziehen sich an (magnetostatisches Grundgesetz)

\subsection*{Historisch}

So um 13/14 Jhd n. Chr. hat man festgestellt, dass sich eine frei drehbare Eisenmodel immer mit einer Spitze in Richtung des geographischen Nordpols ausrichtet.

\subsection*{Merke 1}

Damit ein Stoff magnetische Eigenschaften besitzt, muss er zwingend eines der folgenden drei Metalle enthalten: Eisen, Cobalt oder Nickel.

$downarrow$

\subsection*{Merke 2}

\includegraphics[width=0.4\textwidth]{Magnetisieren.jpg}
\includegraphics[width=0.6\textwidth]{Elementarmagnet.jpg}

\noindent Man kommt mit dem Zerteilen an ein Punkt, ab dem bei einer weiteren Teilung die magnetische Eigenschaft verloren geht $\leftarrow$ Modell des \textbf{Elementarmagneten}

\subsection*{Merke 2}

\textbf{Es gibt keine magnetischen Monopole}

\subsection*{Merke 3}

Die elektrische anziehungskraft ist bei einem Stabmagneten jeweils an den Enden sehr groß, in der Mitte annährend 0.

\subsection*{Das Magnetfeld}

Nähert sich ein Magnet dem anderen, so bewegt sich der andere Magnet nach rechts, ohne dass er berührt wird.

$\rightarrow$ Es muss ein für uns nicht sichtbares Kraftfeld vorhanden sein. Wir nennen es \textbf{Magnetfeld}.

\end{document}