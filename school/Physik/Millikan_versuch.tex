\documentclass{article} 

\usepackage{german}
\usepackage{ mathtools, amssymb, amsthm }
\usepackage{graphicx}
\usepackage{color}
\usepackage{siunitx}

\begin{document}

\section*{Millikan Versuch}

Bei der Auswertung vieler Versuche, stellte Millikan fest, dass die Ladung der Öltröpfchen immer nur als ein ganzzaliges Vielfaches einer kleinsten Ladungsmenge auftreten, und dass es dazwischen keine Ladungen gibt. Man spricht heute von der sogenanten quantelung der Ladung.

$= Ladung Q = n * q		n \widehat{=} \mathbb{N}$

Millikan nannte diese Elementarladung $e$ mit $q_e = 1,602*10^{-19}C$

\includegraphics*{millikan.png}

\end{document}